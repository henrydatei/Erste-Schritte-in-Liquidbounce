Alts steht für \textit{alternative Accounts}. In LiquidBounce gibt es einen recht komfortablen Manager, mit dem man seinen Account wechseln kann, ohne dabei Minecraft neu starten zu müssen.

Man kann sich alternative Accounts im Internet besorgen, Quellen für Alts sind z.B.
\begin{itemize}
    \item FreeGG (nicht zu empfehlen, sehr aufwendig)
    \item Cracker / WareZ / Hacking - Foren (recht gute Ausbeute)
    \item YouTube-Videos (funktionieren aber sehr schlecht)
\end{itemize}

Um die Accounts auf ihre Funktionsfähigkeit zu überprüfen, sollte man beachten, dass das schnelle Ausprobieren von Accounts hintereinander von der selben IP von Mojang schnell mit einem IP-Ban bestraft wird. In diesem Fall heißt es etwa 30 Minuten (?) warten. Es gibt im Internet viele Account-Checker, die diesen Auftrag für einen übernehmen (siehe zum Beispiel \url{https://www.nulled.to} für Alt-Checker). Die meisten Checker kosten allerdings Geld, gecrackte Versionen können allerdings Malware enthalten oder die funktionierenden Accounts an den Entwickler heimlich weiterreichen. Überprüfung mittels Virustotal\footnote{\url{https://www.virustotal.com/\#/home/upload}} auf Viren und Überwachung des Netzwerktraffics mittels Wireshark \footnote{\url{https://www.wireshark.org}} wird empfohlen.

Mit entsprechenden Skripting-Kenntnissen kann man sich aber auch seinen eigenen Account-Checker bauen. Für Linux-User ist folgendes Github-Repository (\url{https://github.com/air/minecraft-tools}) ein guter Anlaufpunkt.

\textbf{ACHTUNG:} Die meisten Alts, die man im Internet bekommt, sind gekaperte / gehackte Minecraft-Konten. Man muss sich beim verwenden solcher Accounts bewusst sein, dass man damit das Geschäft mit gecrackten Accounts ankurbelt. Während hacken in Spielen völlig legal ist (wenn auch unerwünscht von Einigen), sind Alts daher je nach Quelle entweder als Grauzone oder als illegal einzustufen.