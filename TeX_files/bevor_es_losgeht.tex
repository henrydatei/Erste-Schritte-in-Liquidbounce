LiquidBounce ist ein Minecraft Hackclient. Man wird mit diesem Client - bei richtiger Benutzung und vielleicht etwas Übung - viele Vorteile gegenüber anderen Spielern in Minecraft haben.

Auch wenn immer die Rede von \textit{Hackclients} oder \textit{Hacks} ist, gehackt wird im eigentlichen Sinne nicht. Vielmehr ist es ein Manipulieren von Paketen, die man zum Server sendet, also fällt es in die Kategorie des \textit{Cheating}. Der Einfachheit halber spreche ich in diesem Dokument von \textit{Hacks}.

Es sollte klar sein, dass Legits und Serverbesitzer den Einsatz eines Clients nicht gutheißen. Legits beleidigen gerne und Serverbesitzer bannen gerne. Gegen beides gibt es aber Möglichkeiten vorzugehen:
\begin{enumerate}[label=(\alph*)]
    \item Den Chat ausschalten (Options $\to$ Chat Settings $\to$ Chat: Hidden)
    \item alternative Accounts (Alts) und eventuell ein VPN
\end{enumerate}
In den meisten Fällen macht es aber Spaß mit den Legits zu schreiben und sie zur Verzweiflung und zum Rage-Quit zu bringen.