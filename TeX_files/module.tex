LiquidBounce hat sehr viele Module und es ist schwer da am Anfang einen Überblick zu bekommen. Ich habe deswegen die wichtigsten Module \textcolor{lime!50}{grün} hinterlegt.

\subsection{Exploit}

Die folgenden Module sind Exploits, die in der Regel auf bestimmten Versionen von Anticheats funktionieren, aber in aller Regel nicht.
\begin{longtable}{p{3cm}|p{10cm}}
\textbf{Hack} & \textbf{Erklärung} \\
\hline
AbortBreaking & Man kann das Abbauen eines Blockes unterbrechen und später weitermachen, ohne seinen Fortschritt zu verlieren. \\
\hline
AntiHunger & Funktioniert manchmal auf alten AAC-Versionen und unterdrückt, dass man Hunger bekommt. \\
\hline
BedGodMode & Funktioniert nur in der 1.9 und ist damit in einem 1.8-Client sinnlos. \\
\hline
ConsoleSpammer & Spammt die Serverkonsole mit Müll zu. Man wird dabei in der Regel gekickt. \\
\hline
Damage & Man fügt sich selber Schaden zu. \\
\hline
EmptyMessage & Sendet eine leere Nachricht im Chat. \\
\hline
ForceUnicodeChat & Ersetzt alle ASCII-Zeichen im Chat durch Unicode-Zeichen. \\
\hline
Ghost & Nachdem man gestorben ist, kann man weiter herumlaufen. Ich sehe keine Verwendung dafür. \\
\hline
GhostHand & Erlaubt es mit Blöcken durch die Wand zu interagieren. Recht nützlich bei Kisten. Man muss den Block, mit dem man interagieren möchte mit \texttt{.ghosthand select chest} einstellen. \\
\hline
GodMode & Ein Exploit für eine alte Version für AAC mit der man nicht sterben konnte. \\
\hline
ItemTeleport & Man kann sich zu Items teleportieren, funktioniert in der Regel nicht. \\
\hline
KeepContainer & Man kann Container (Kisten, Villager-Inventare, ...) von überall öffnen. Noch nie benutzt ... \\
\hline
Kick & Man kann sich selber kicken. Nützlich, wenn man gerade einen Kampf verliert, aber bestraft wird, wenn man den Server verlässt. Ähnliche Funktion wie Damage. \\
\hline
MoreCarry & Man kann in den 4 Crafting-Slots im Inventar Items lagern. \\
\hline
MultiActions & Man kann andere Items benutzen, wenn man gerade Blöcke abbaut. \\
\hline
NoPitchLimit & keine Ahnung ... \\
\hline
Paralyze & Wenn man sich in andere Spieler reinstellt, fängt deren Game an zu laggen. Aber leider auch dein Eigenes ... \\
\hline
Phase & Man kann durch Blöcke laufen. Ähnlich wie NoClip. \\
\hline
PingSpoof & Setzt den vom Server \textit{gesehenen Ping} auf einen festgelegten Wert. Kann nützlich sein, um Items als erster zu bekommen, oder bei manchen Servern Anti-Knockback zu deaktivieren. \\
\hline
\rowcolor{lime!50}Plugins & Zeigt die Plugins an, die auf einem Server installiert sind. Funktioniert recht häufig. \\
\hline
ServerCrasher & Name ist Programm ... Zumindest manchmal. \\
\hline
Teleport & Wenn aktiviert, muss man mit der mittleren Maustaste (in der Regel dem Mausrad) einen Block auswählen. Mit SHIFT teleportiert man sich dann dahin. Nützlich für Jump-and-Runs in Lobbys. \\
\hline
VClip & Teleportiert einen senkrecht nach oben. \\
\hline
VehicleOneHit & Man kann Minecarts mit einem Schlag zerstören. \\
\end{longtable}

\subsection{World}

\begin{longtable}{p{3cm}|p{10cm}}
\textbf{Hack} & \textbf{Erklärung} \\
\hline
AutoBreak & Baut den Block ab, den man gerade ansieht. \\
\hline
\rowcolor{lime!50}ChestAura & Öffnet Kisten in einem Radius. Nützlich in Zusammenhang mit dem ChestStealer. \\
\hline
\rowcolor{lime!50}ChestStealer & Wenn man eine Kiste öffnet, werden alle Inhalte automatisch ins Inventar gezogen, und die Kiste wieder geschlossen. Bei Survival-Games äußerst praktisch. Bitte in Lobbys deaktivieren, da ansonsten deren Menüs buggen. \\
\hline
CivBreak & Baut Blöcke instant ab. Funktioniert in der Regel nicht. \\
\hline
FastBreak & Baut Blöcke schneller ab, sorgt aber in der Regel für Flags, bringt also keinen Geschwindigkeitsvorteil. \\
\hline
FastPlace & Man kann Blöcke deutlich schneller platzieren. Benutzen Legits auch häufig. \\
\hline
\rowcolor{lime!50}Fucker & Zerstört automatisch einen festgelegten Blocktyp in einem Radius um den Spieler (teils auch durch Wände). Macht einen zum Bedwars- / Cakewars-King. Der zu zerstörende Block muss aber mit \texttt{.fucker select bed} eingestellt werden. \\
\hline
Liquids & Erlaubt es einem mit Flüssigkeiten zu interagieren, das heißt auf Wasserquellen Blöcke oder andere Wasserquellen zu setzen. \\
\hline
Nuker & Baut in einem Ring alle Blöcke ab. Vorsicht im Creative-Modus, die Welt ist schnell kaputt. \\
\hline
\rowcolor{lime!50}Scaffold & Setzt Blöcke vor dem Spieler ins Void. Nützlich um sich bei Bedwars rüber zu bauen. \\
\hline
Timer & Beschleunigt die Zeit in Minecraft. Kann als Speed benutzt werden, wird aber häufig erkannt. \\
\hline
\rowcolor{lime!50}Tower & Stackt den Spieler hoch, und zwar extrem schnell. \\
\end{longtable}

\subsection{Targets}

Das sind eigentlich Einstellungen und keine Hacks. Sie beeinflussen, welche Ziele die Killaura angreift und welche Entities im ESP angezeigt werden.

\begin{longtable}{p{3cm}|p{10cm}}
\textbf{Einstellung} & \textbf{Erklärung} \\
\hline
Players & normale Spieler \\
\hline
Mobs & Bösartige Wesen (Zombies, Skeletons etc.) \\
\hline
Animals & Tiere \\
\hline
Invisible & Unsichtbare Wesen / Spieler. \\
\hline
Dead & Tote Spieler / Wesen. Bei einigen Anticheats benötigt, um Spieler korrekt zu erkennen. \\
\end{longtable}

\subsection{Render}

Render-Hacks können von Anticheats nicht erkannt werden. Das ist nur durch Verhaltensanalyse, zum Beispiel bei häufigem schnellen Abbau von Diamanten hintereinander auf Survival-Servers $\to$ X-Ray-Hack.

\begin{longtable}{p{3cm}|p{10cm}}
\textbf{Hack} & \textbf{Erklärung} \\
\hline
\rowcolor{lime!50}AntiBlind & Entfernt den Blindness- und den Nausea-Effekt. \\
\hline
\rowcolor{lime!50}BlockESP & Man sieht den eingestellten Block durch die Wand, zum Beispiel ein Bett bei Bedwars. Muss mit \texttt{.blockesp select bed} konfiguriert werden. \\
\hline
BlockOverlay & Ein anderes Block Overlay. Finde ich hässlich. \\
\hline
Breadcrumbs & Zeichnet eine Linie überall wo man langgeht. \\
\hline
CameraClip & Man kann in der Third-Person-View durch Wände sehen. Man kann aber auch einen ESP benutzen. \\
\hline
Chams & Siehe ESP. Braucht man eigentlich nicht. \\
\hline
\rowcolor{lime!50}ClickGUI & Die Übersicht über alle Hacks. \\
\hline
\rowcolor{lime!50}ESP & Ich kann sehen was du nicht siehst! Durch Wände. Einer der wichtigsten und grundlegendsten Hacks, um sich einen Spielvorteil zu verschaffen. Sollte immer angeschaltet sein. \\
\hline
FreeCam & Ist sie aktiviert, kann man frei herumfliegen mit einer virtuellen Kamera. Man kann also auch sich selbst betrachten. \\
\hline
\rowcolor{lime!50}Fullbright & Man sieht die Welt immer wie am Tag. Sollte immer an sein. \\
\hline
HUD & Damit kann man konfigurieren, welche Elemente man sehen möchte, wenn man nicht in der ClickGUI ist. \\
\hline
HeadRotations & Man sieht die Richtung der Köpfe wie der Server sie sieht. Sinnlos. \\
\hline
\rowcolor{lime!50}ItemESP & Wie ESP, aber für Items. \\
\hline
\rowcolor{lime!50}NameTags & Wichtige Ergänzung für den ESP: Neben den Spielernamen werden Daten dieser Spieler angezeigt wie Leben und Equipment, sowie Rüstung. Macht aber auch die Namen über den Spielern größer und Map-weit sichtbar. \\
\hline
NoBob & Verhindert irgendeinen Clientside-Effekt. Kann man anschalten, muss man nicht. \\
\hline
NoFOV & Verhindert den Zoom-Effekt bei Speed-Potions und beim Sprinten. \\
\hline
NoHurtCam & Bild wackelt nicht, wenn man angegriffen wird. \\
\hline
NoScoreboard & Kein Scoreboard mehr \\
\hline
NoSwing & Keine Swing-Animation des Schwertes. \\
\hline
Projectiles & Man sieht, wo Pfeile, Enderperlen, etc. landen werden. \\
\hline
ProphuntESP & Ein ESP für Versteck-Spielmodi. Funktioniert aber nicht immer richtig. \\
\hline
RemoteView & Erlaubt es das Gesehen durch einen anderen Spieler zu sehen. \texttt{.remoteview Spielername} \\
\hline
\rowcolor{lime!50}StorageESP & Kisten sind durch Wände zu sehen. \\
\hline
SwingAnimation & Eine andere Schwung-Animation des Schwertes. \\
\hline
Tracers & Linien vom Fadenkreuz zu den nächsten Spielern. \\
\hline
TrueSight & Man sieht unsichtbare Entities, wie im Gamemode 3. \\
\hline
\rowcolor{lime!50}XRay & Alle Blöcke außer Erzen, Kisten, etc. sind transparent. \\
\end{longtable}

\subsection{Movement}

\begin{longtable}{p{3cm}|p{10cm}}
\textbf{Hack} & \textbf{Erklärung} \\
\hline
AirJump & Wie ein Double Jump, nur, dass man nicht nur zwei Mal, sondern immer wieder springen kann. Kommt somit einem Fly sehr nahe. \\
\hline
AirLadder & Man kann Leitern hochklettern, ohne sie zu berühren. Hab ichnoch nie benutzt. \\
\hline
AutoWalk & Läuft automatisch in die Richtung, in die man schaut. \\
\hline
BlockWalk & Man kann auf Halfslabs laufen. Noch nie benutzt. \\
\hline
BufferSpeed & Man bekommt auf bestimmten Blöcken einen Speed-Boost. \\
\hline
BugUp & Fällt man im freien Fall (zum Beispiel in Bedwars), so versucht dieser Hack, einen wieder auf einen Block zu teleportieren. Geht eher selten, kann aber nützlich sein. \\
\hline
FastClimb & Leitern schneller hochklettern. \\
\hline
FastStairs & Treppen schneller hochsteigen. \\
\hline
\rowcolor{lime!50}Fly & Durch die Welt fliegen, wie im Creative. Sehr einfach von Moderatoren \textit{und} Anticheat zu erkennen. \\
\hline
Freeze & Man kann in der Luft stehen bleiben. Keine Ahnung wozu das gut sein soll. \\
\hline
HighJump & höher springen \\
\hline
IceSpeed & schneller auf Eis laufen \\
\hline
\rowcolor{lime!50}InventoryMove & Man kann sich mit geöffnetem Inventar (und kann dort sogar sortieren und craften) bewegen. Funktioniert eigentlich immer. \\
\hline
LadderJump & Leitern geben einen Boost nach oben. \\
\hline
LiquidWalk & auch Jesus-Hack. Man läuft auf Wasser und Lava. Manchmal bekommt man sogar keinen Schaden von der Lava. \\
\hline
LongJump & weiter springen \\
\hline
NoClip & Man kann durch Wände laufen. Klappt in der Regel nicht. \\
\hline
NoJumpDelay & Kein Delay beim Springen. Hab da aber keinen Unterschied gespürt. \\
\hline
\rowcolor{lime!50}NoSlow & Man wird nicht durch Blöcke (wie Seelensand) oder Potions verlangsamt. \\
\hline
\rowcolor{lime!50}NoWeb & Wie NoSlow, bei Spinnennetzen \\
\hline
Parkour & Springt an Blockkanten automatisch so weit wie möglich, was bei manchen Jump'n'Runs nützlich ist. \\
\hline
PerfectHorseJump & lässt ein Pferd immer am höchsten springen \\
\hline
ReverseStep & Zieht einen schnell nach unten. Sehr gewöhnungsbedürftig. \\
\hline
SafeWalk & Man muss nicht mehr sneaken, um an Blockkanten nicht mehrrunterzufallen. \\
\hline
SlimeJump & Man springt auf Slime-Blöcken höher. \\
\hline
Sneak & Man sneakt immer. \\
\hline
\rowcolor{lime!50}Speed & Immer am schnellsten unterwegs sein. Verhält sich je nach Modus sehr verschieden, von BHop bis Gleiten ist alles dabei. Ein guter Hacker toggled den Speedhack nur und lässt ihn keinesfalls immer an. Durch Anticheats oft recht einfach zu erkennen. \\
\hline
\rowcolor{lime!50}Sprint & Drückt automatisch Doppel-W. Immer anlassen, außer wenn man hierdurch zu viel Essen verliert. Klappt auf manchen Servern auch rückwärts. \\
\hline
\rowcolor{lime!50}Step & Springt an Blockkanten automatisch hoch, teleportiert dich aber in der Regel den Block hoch. Bei manchen Servern sogar an Kanten, welche für Legits zu hoch sind (1,5 - 2 Blöcke hoch). \\
\hline
Strafe & Man kann extremer Strafen im Sprung (also zum Beispiel einen 360 machen, Projektilen ausweichen) \\
\hline
WallClimb & Wände hochklettern, wie wenn Leitern daran wären. \\
\hline
WallGlide & Man kann an Wänden gleiten. \\
\hline
WaterFly & Wie fliegen, aber nur unter Wasser. \\
\hline
WaterSpeed & Schneller im Wasser unterwegs sein. \\
\end{longtable}

\subsection{Player}

\begin{longtable}{p{3cm}|p{10cm}}
\textbf{Hack} & \textbf{Erklärung} \\
\hline
AntiAFK & Umgeht AFK - Timeouts. \\
\hline
\rowcolor{lime!50} AntiCactus & Keinen Schaden durch Kakteen bekommen. Sollte immer an sein. \\
\hline
AutoFish & Automatisch Fischen \\
\hline
\rowcolor{lime!50} AutoRespawn & Automatisch Respawnen \\
\hline
\rowcolor{lime!50} AutoTool & Verwendet immer das beste Tool. Somit kann man mit einem Schwert Blöcke im Kampf abbauen, da während des Abbauens automatisch z.B. eine Spitzhacke genutzt wird. \\
\hline
Blink & Stoppt die Kommunikation mit dem Server um einen Lag zu simulieren. Beim Deaktivieren wird die Kommunikation wieder freigegeben. Sieht für andere Spieler wie eine Teleportation aus. Wird aber recht zuverlässig von Anticheats erkannt. \\
\hline
Eagle & Scaffold für Legits. Braucht man nicht, Scaffold funktioniert eigentlich immer. \\
\hline
FarmKingBot & irgendein Bot für einen Spielmodus \\
\hline
\rowcolor{lime!50} FastUse & Schneller Essen. Wenn es geht sehr nützlich, aber ein oft vergessener Grund für unerklärliche Banns, wenn es nicht geht. \\
\hline
\rowcolor{lime!50} InventoryCleaner & Sortiert und reinigt das Inventar, legt das Schwert auf Slot 1. \\
\hline
\rowcolor{lime!50} NoFall & Kein Fallschaden. \\
\hline
PotionSaver & Wenn man still steht, läuft die Zeit von Effekten nicht ab. \\
\hline
Reach & Höhere Reichweite \\
\hline
Regen & Leben aufladen. Geht nur auf Vanilla-Servern. Auch dort darf man sich währenddessen nicht in der Luft sein (kann aber per Option auf in der Luft automatisch deaktiviert werden). Also nur mit Toggle nutzen. \\
\hline
Zoot & Keine negativen Potion-Effekte. Geht nur äußerst selten und bannt recht häufig. \\
\end{longtable}

\subsection{Fun}

\begin{longtable}{p{3cm}|p{10cm}}
\textbf{Hack} & \textbf{Erklärung} \\
\hline
Derp & keine Ahnung, noch nie benutzt \\
\hline
SkinDerp & keine Ahnung, noch nie benutzt \\
\end{longtable}

\subsection{Combat}

\begin{longtable}{p{3cm}|p{10cm}}
\textbf{Hack} & \textbf{Erklärung} \\
\hline
Aimbot & zieht das Fadenkreuz auf den nächsten Gegner. Kann mit Trigger zum Killaura-Ersatz ausgebaut werden. \\
\hline
\rowcolor{lime!50} AutoArmor & Nutzt automatisch die beste Rüstung im Inventar. \\
\hline
AutoBow & Schießt automatisch mit dem Bogen, wenn dieser voll gespannt ist. \\
\hline
AutoClicker & Klickt automatisch. \\
\hline
AutoLeave & Verlässt den Server, wenn man unter einer bestimmten Anzahl an Leben ist. Siehe dazu auch Damage und Kick. \\
\hline
\rowcolor{lime!50} AutoPot & Wirft automatisch eine Health-Potion auf dich, wenn du unter einer bestimmten Anzahl an Leben bist. \\
\hline
\rowcolor{lime!50} AutoSoup & Isst eine Suppe, wenn man unter einer bestimmten Anzahl von Leben ist. Nützlich bei Soup-PVP-Kämpfen. \\
\hline
AutoSword / AutoWeapon & Nimmt automatisch das beste Schwert in der Hotbar zum Angreifen. \\
\hline
\rowcolor{lime!50} BowAimbot & Aimbot für den Bogen, wie man ihn von Shootern kennt. Der Silent-Modus ist unnötig, den kann man getrost ausschalten. \\
\hline
\rowcolor{lime!50} Criticals & Verteilt immer den maximalen Schaden mit einigen Kniffen, zum Beispiel indem es beim Attackieren springt oder es sendet ein entsprechendes Packet. \\
\hline
FastBow & Pfeile spammen \\
\hline
\rowcolor{lime!50} Hitbox & Hitbox vergrößern. Undetectable bzw. wurde dafür noch nie gebannt. \\
\hline
Ignite & Zündet automatisch Gegner mit einem Feuerzeug an. \\
\hline
\rowcolor{lime!50} Killaura & Das heilige Gral des Hackers. Verteilt Spielern in einem gewissen Radius vollautomatisch Schaden, und funktioniert dabei ähnlich wie eine Kombination aus Triggerbot und Aimbot. \\
\hline
NoFriends & Normalerweise werden Leute in der Friendlist (\texttt{.friend add NAME} / blau angezeigt im ESP) nicht angegriffen. Mit NoFriends deaktiviert man das, zum Beispiel für Duelle gegen Freunde. \\
\hline
SuperKnockback & Durch Sprinten vor einem Schlag wird der Knockback des Gegners maximiert. Undetectable. \\
\hline
TNTBlock & Blockt automatisch mit dem Schwert, wenn TNT in der Nähe explodiert. \\
\hline
TeleportHit & Erlaubt es bei manchen Anticheats Gegner aus großer Entfernung zu schlagen. \\
\hline
Trigger & Schlägt automatisch, wenn man mit dem Fadenkreuz auf einem Gegner ist. \\
\hline
\rowcolor{lime!50} Velocity & Kein Knockback mehr, oder zumindest zu wenig wie möglich. Kann man eigentlich immer an lassen, aber ist auch von ungeschulten Personen sehr leicht zu erkennen. \\
\end{longtable}

\subsection{Misc}
\begin{longtable}{p{3cm}|p{10cm}}
\textbf{Hack} & \textbf{Erklärung} \\
\hline
\rowcolor{lime!50} AntiBot & Verhindert, dass die Killaura Fake-Spieler angreift, die versuchen eine Killaura zu erkennen. \\
\hline
AtAllProvider & keine Ahnung \\
\hline
ControllerControl & Erlaubt es Minecraft mit dem XBOX-Controller zu steuern. \\
\hline
LiquidChat & Ein alternativer Chat zwischen LiquidBounce Nutzern. \\
\hline
MidClick & Aktion beim Mittelklick festlegen, zum Beispiel Leute automatisch als Freunde adden. \\
\hline
NameProtect & Verändert alle Namen der Mitspieler. Wichtig bei Streamern, dass diese nicht so schnell gefunden werden können. \\
\hline
NoRotateSet & Verhindert, dass der Server deinen Kopf dreht (das passiert häufig beim Scaffold). Funktioniert nicht immer. \\
\hline
Spammer & Spammt den Chat zu, mit einer einstellbaren Zeile. \\
\hline
\rowcolor{lime!50} Teams & Erkennt automatisch das eigene Team, und attackiert nur die anderen Teams. \\
\end{longtable}