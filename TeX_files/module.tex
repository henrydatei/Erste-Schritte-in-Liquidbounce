LiquidBounce hat sehr viele Module und es ist schwer da am Anfang einen Überblick zu bekommen. Ich habe deswegen die wichtigsten Module \textcolor{lime!50}{grün} hinterlegt.

\subsection{Exploit}

Die folgenden Module sind Exploits, die in der Regel auf bestimmten Versionen von Anticheats funktionieren, aber in aller Regel nicht.
\begin{longtable}{p{3cm}|p{10cm}}
\textbf{Hack} & \textbf{Erklärung} \\
\hline
AbortBreaking & Man kann das Abbauen eines Blockes unterbrechen und später weitermachen, ohne seinen Fortschritt zu verlieren. \\
\hline
AntiHunger & Funktioniert manchmal auf alten AAC-Versionen und unterdrückt, dass man Hunger bekommt. \\
\hline
BedGodMode & Funktioniert nur in der 1.9 und ist damit in einem 1.8-Client sinnlos. \\
\hline
ConsoleSpammer & Spammt die Serverkonsole mit Müll zu. Man wird dabei in der Regel gekickt. \\
\hline
Damage & Man fügt sich selber Schaden zu. \\
\hline
EmptyMessage & Sendet eine leere Nachricht im Chat. \\
\hline
ForceUnicodeChat & Ersetzt alle ASCII-Zeichen im Chat durch Unicode-Zeichen. \\
\hline
Ghost & Nachdem man gestorben ist, kann man weiter herumlaufen. Ich sehe keine Verwendung dafür. \\
\hline
GhostHand & Erlaubt es mit Blöcken durch die Wand zu interagieren. Recht nützlich bei Kisten. Man muss den Block, mit dem man interagieren möchte mit \texttt{.ghosthand select chest} einstellen. \\
\hline
GodMode & Ein Exploit für eine alte Version für AAC mit der man nicht sterben konnte. \\
\hline
ItemTeleport & Man kann sich zu Items teleportieren, funktioniert in der Regel nicht. \\
\hline
KeepContainer & Man kann Container (Kisten, Villager-Inventare, ...) von überall öffnen. Noch nie benutzt ... \\
\hline
Kick & Man kann sich selber kicken. Nützlich, wenn man gerade einen Kampf verliert, aber bestraft wird, wenn man den Server verlässt. Ähnliche Funktion wie Damage. \\
\hline
MoreCarry & Man kann in den 4 Crafting-Slots im Inventar Items lagern. \\
\hline
MultiActions & Man kann andere Items benutzen, wenn man gerade Blöcke abbaut. \\
\hline
NoPitchLimit & keine Ahnung ... \\
\hline
Paralyze & Wenn man sich in andere Spieler reinstellt, fängt deren Game an zu laggen. Aber leider auch dein Eigenes ... \\
\hline
Phase & Man kann durch Blöcke laufen. Ähnlich wie NoClip. \\
\hline
PingSpoof & Setzt den vom Server \textit{gesehenen Ping} auf einen festgelegten Wert. Kann nützlich sein, um Items als erster zu bekommen, oder bei manchen Servern Anti-Knockback zu deaktivieren. \\
\hline
\rowcolor{lime!50}Plugins & Zeigt die Plugins an, die auf einem Server installiert sind. Funktioniert recht häufig. \\
\hline
ServerCrasher & Name ist Programm ... Zumindest manchmal. \\
\hline
Teleport & Wenn aktiviert, muss man mit der mittleren Maustaste (in der Regel dem Mausrad) einen Block auswählen. Mit SHIFT teleportiert man sich dann dahin. Nützlich für Jump-and-Runs in Lobbys. \\
\hline
VClip & Teleportiert einen senkrecht nach oben. \\
\hline
VehicleOneHit & Man kann Minecarts mit einem Schlag zerstören. \\
\end{longtable}

\subsection{World}

\begin{longtable}{p{3cm}|p{10cm}}
\textbf{Hack} & \textbf{Erklärung} \\
\hline
AutoBreak & Baut den Block ab, den man gerade ansieht. \\
\hline
\rowcolor{lime!50}ChestAura & Öffnet Kisten in einem Radius. Nützlich in Zusammenhang mit dem ChestStealer. \\
\hline
\rowcolor{lime!50}ChestStealer & Wenn man eine Kiste öffnet, werden alle Inhalte automatisch ins Inventar gezogen, und die Kiste wieder geschlossen. Bei Survival-Games äußerst praktisch. Bitte in Lobbys deaktivieren, da ansonsten deren Menüs buggen. \\
\hline
CivBreak & Baut Blöcke instant ab. Funktioniert in der Regel nicht. \\
\hline
FastBreak & Baut Blöcke schneller ab, sorgt aber in der Regel für Flags, bringt also keinen Geschwindigkeitsvorteil. \\
\hline
FastPlace & Man kann Blöcke deutlich schneller platzieren. Benutzen Legits auch häufig. \\
\hline
\rowcolor{lime!50}Fucker & Zerstört automatisch einen festgelegten Blocktyp in einem Radius um den Spieler (teils auch durch Wände). Macht einen zum Bedwars- / Cakewars-King. Der zu zerstörende Block muss aber mit \texttt{.fucker select bed} eingestellt werden. \\
\hline
Liquids & Erlaubt es einem mit Flüssigkeiten zu interagieren, das heißt auf Wasserquellen Blöcke oder andere Wasserquellen zu setzen. \\
\hline
Nuker & Baut in einem Ring alle Blöcke ab. Vorsicht im Creative-Modus, die Welt ist schnell kaputt. \\
\hline
\rowcolor{lime!50}Scaffold & Setzt Blöcke vor dem Spieler ins Void. Nützlich um sich bei Bedwars rüber zu bauen. \\
\hline
Timer & Beschleunigt die Zeit in Minecraft. Kann als Speed benutzt werden, wird aber häufig erkannt. \\
\hline
\rowcolor{lime!50}Tower & Stackt den Spieler hoch, und zwar extrem schnell. \\
\end{longtable}

\subsection{Targets}

Das sind eigentlich Einstellungen und keine Hacks. Sie beeinflussen, welche Ziele die Killaura angreift und welche Entities im ESP angezeigt werden.

\begin{longtable}{p{3cm}|p{10cm}}
\textbf{Einstellung} & \textbf{Erklärung} \\
\hline
Players & normale Spieler \\
\hline
Mobs & Bösartige Wesen (Zombies, Skeletons etc.) \\
\hline
Animals & Tiere \\
\hline
Invisible & Unsichtbare Wesen / Spieler. \\
\hline
Dead & Tote Spieler / Wesen. Bei einigen Anticheats benötigt, um Spieler korrekt zu erkennen. \\
\end{longtable}

\subsection{Render}

Render-Hacks können von Anticheats nicht erkannt werden. Das ist nur durch Verhaltensanalyse, zum Beispiel bei häufigem schnellen Abbau von Diamanten hintereinander auf Survival-Servers $\to$ X-Ray-Hack.

\begin{longtable}{p{3cm}|p{10cm}}
\textbf{Hack} & \textbf{Erklärung} \\
\hline
\rowcolor{lime!50}AntiBlind & Entfernt den Blindness- und den Nausea-Effekt. \\
\hline
\rowcolor{lime!50}BlockESP & Man sieht den eingestellten Block durch die Wand, zum Beispiel ein Bett bei Bedwars. Muss mit \texttt{.blockesp select bed} konfiguriert werden. \\
\hline
BlockOverlay & Ein anderes Block Overlay. Finde ich hässlich. \\
\hline
Breadcrumbs & Zeichnet eine Linie überall wo man langgeht. \\
\hline
CameraClip & Man kann in der Third-Person-View durch Wände sehen. Man kann aber auch einen ESP benutzen. \\
\hline
Chams & Siehe ESP. Braucht man eigentlich nicht. \\
\hline
\rowcolor{lime!50}ClickGUI & Die Übersicht über alle Hacks. \\
\hline
\rowcolor{lime!50}ESP & Ich kann sehen was du nicht siehst! Durch Wände. Einer der wichtigsten und grundlegendsten Hacks, um sich einen Spielvorteil zu verschaffen. Sollte immer angeschaltet sein. \\
\hline
FreeCam & Ist sie aktiviert, kann man frei herumfliegen mit einer virtuellen Kamera. Man kann also auch sich selbst betrachten. \\
\hline
\rowcolor{lime!50}Fullbright & Man sieht die Welt immer wie am Tag. Sollte immer an sein. \\
\hline
HUD & Damit kann man konfigurieren, welche Elemente man sehen möchte, wenn man nicht in der ClickGUI ist. \\
\hline
HeadRotations & Man sieht die Richtung der Köpfe wie der Server sie sieht. Sinnlos. \\
\hline
\rowcolor{lime!50}ItemESP & Wie ESP, aber für Items. \\
\hline
\rowcolor{lime!50}NameTags & Wichtige Ergänzung für den ESP: Neben den Spielernamen werden Daten dieser Spieler angezeigt wie Leben und Equipment, sowie Rüstung. Macht aber auch die Namen über den Spielern größer und Map-weit sichtbar. \\
\hline
NoBob & Verhindert irgendeinen Clientside-Effekt. Kann man anschalten, muss man nicht. \\
\hline
NoFOV & Verhindert den Zoom-Effekt bei Speed-Potions und beim Sprinten. \\
\hline
NoHurtCam & Bild wackelt nicht, wenn man angegriffen wird. \\
\hline
NoScoreboard & Kein Scoreboard mehr \\
\hline
NoSwing & Keine Swing-Animation des Schwertes. \\
\hline
Projectiles & Man sieht, wo Pfeile, Enderperlen, etc. landen werden. \\
\hline
ProphuntESP & Ein ESP für Versteck-Spielmodi. Funktioniert aber nicht immer richtig. \\
\hline
RemoteView & Erlaubt es das Gesehen durch einen anderen Spieler zu sehen. \texttt{.remoteview Spielername} \\
\hline
\rowcolor{lime!50}StorageESP & Kisten sind durch Wände zu sehen. \\
\hline
SwingAnimation & Eine andere Schwung-Animation des Schwertes. \\
\hline
Tracers & Linien vom Fadenkreuz zu den nächsten Spielern. \\
\hline
TrueSight & Man sieht unsichtbare Entities, wie im Gamemode 3. \\
\hline
\rowcolor{lime!50}XRay & Alle Blöcke außer Erzen, Kisten, etc. sind transparent. \\
\end{longtable}

\subsection{Movement}

\subsection{Player}

\subsection{Fun}

\subsection{Combat}

\subsection{Misc}