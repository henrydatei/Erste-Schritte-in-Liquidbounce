Befehle beginnen in LiquidBounce mit einem Punkt \texttt{.}. Damit kann mal alle Einstellungen der Module auch ohne die ClickGUI machen. Benutzt wird dieses Feature unter anderem für die Autosettings - damit wird der Client gleich mit den passenden Bypasses für den Server ausgerüstet.

Um die Autosettings eines Servers zu laden, gibt man in de Chat ein: \texttt{.settings load gomme}. Autosettings sind für viele Server verfügbar, eine Liste bekommt man mit \texttt{.settings list} oder man schaut einfach auf der Github-Seite des Entwicklerteams nach: \url{https://github.com/CCBlueX/FileCloud/tree/master/LiquidBounce/autosettings}

Weitere nützliche Befehle:
\begin{itemize}
    \item \texttt{.bind Modulname KEY} - Hotkeys für Module festlegen
    \item \texttt{.friends [add/list/remove] Name} - Friendlist verwalten
    \item \texttt{.localsettings [save/list/load] Name} - Eigene Settings abspeichern
    \item \texttt{.serverinfo} - Infos über aktuellen Server
    \item \texttt{.help} - Zeigt Hilfe an
\end{itemize}